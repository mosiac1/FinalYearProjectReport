\chapter{Introduction}
\label{cha:intro}

In the last few decades, the explosion of digital technology has led to an increasing amount of data being collected and generated. The rise of big data has led to many opportunities and challenges for organisations and individuals alike. However, with this abundance of data comes the risk of privacy infringement and the abuse of user information. Metadata has become a critical component in securing individuals' privacy and user information.

Metadata can provide valuable insights into the structure of data and how it is used, but it can also be used to track, identify and monitor individuals without their consent. As such, it is crucial to ensure that system administrators have control over all information about the data their organisation stores and how it is used. This is where metadata authorisation comes into play.

Metadata authorisation is the process of granting or denying access to metadata. It is crucial for any organisation that collects user data to have a robust metadata authorisation system that enables the protection of the privacy and personally-identifying information of individuals while still allowing for the most effective use of data and metadata. Metadata authorisation is crucial in industries such as healthcare, finance, and marketing, where sensitive personal information is frequently collected.

This report explores the various challenges and solutions related to metadata authorisation and discusses the technical challenges of implementing a robust metadata authorisation system on top of industry-standard technologies.

 \section{Background}

 Metadata is "data that provides information about other data", but not directly about the content of the data, such as individual rows in a table or the pixels of an image. Categories of metadata commonly encountered include but are not limited to \cite{understandingMetadataRiley2017}:

\begin{itemize}
    \item Descriptive metadata - relates to finding and comprehending data. As an example, a book may consist of the title, author, subject and ISBN;
    \item Administrative metadata - relates to how the data is intended to be used. Examples of this type are technical metadata (e.g., the format of a file, instructions for decoding), preservation or authenticity metadata (e.g., versioning artefacts, digital signatures, check-sums) and rights of legal metadata (e.g., licensing);
    \item Structural metadata - describing relationships of data to other resources;
    \item Statistical metadata - often used in statistics, describing the procurement, processing and aggregation done to create data sets \cite{roleOfMetadataInStatisticsdippo2000}. 
\end{itemize}

Big data refers to data sets that are too large and complex to be dealt with using traditional data processing methods and software while being at the centre of modern business and science. Until 2003, humans created roughly 5 exabytes ($10^{18}$ bytes) of stored data. In 2013, that amount was generated in two days \cite{bigDataAReviewsagiroglu2013}. Big data is characterised by its three V's \cite{understandingBigDataZikopoulos2011}:

\begin{itemize}
    \item Variety - the structure of data, which comes from many heterogeneous sources in different formats. Data may be structured, when it is warehoused, well organised, tagged and labelled, semi-structured or unstructured, where most data resides, random, difficult and costly to use effectively;
    \item Volume - the size of data, which is always increasing, is larger than the petabyte scale;
    \item Velocity - the speed at which data changes dictates the optimal way to use it. Data may be acquired in batches or arrive as a discrete stream of samples.
\end{itemize}

In the context of Big data, metadata is concerned with what the data is, where it comes from, how it was processed, what is the data type of each of its elements, what units it is measured in, what formulas were used to derive it, who owns it, governance processes. Metadata is a core part of the effective use of big data.

Security is a significant concern in the context of Big Data systems, concerning protecting confidentiality, integrity and data availability. Data centralised in one place opens itself up as a valuable attack target. Big Data privacy is also a relevant matter, concerned with protecting individuals' personal information. These systems commonly store and aggregate large amounts of personally identifying information collected from various sources, so mismanagement can seriously impact the users' privacy \cite{bigDataSecurityTankard20125}. There is an increasing number of regulations organisations handling user data must follow relating to how data is collected and where it is stored and used. For example, the healthcare industry has been adopting the Big Data approach for storing patient records, which are inherently \acrshort{pii} and are heavily regulated, thus privacy in those systems is a major point of contention \cite{bigDataSecurityHealthPatil2014}.

\section{Motivation}

Data is growing exponentially, and soon all data may become Big Data. The Big Data approach has been widely adopted to handle large amounts of data in the computer science world for some time. While not a solved problem, storing large quantities of data has become more manageable with the decrease in the price of storage and processing units and the increase in the popularity of unstructured, file-based storage on distributed systems like AWS's S3, Casandra, and HDFS. Doing large and complex computations on these data sets is extensively aided by the ability to detach computational power from storage, something \acrshort{dbms} are limited by, with \gls{data_mesh} tools such as Trino\footnote{Formerly known as Presto} \cite{trinoTech, trinoSethi2019}, Dremio \cite{dremioTech}, or Apache Drill \cite{apacheDrillTech, apacheDrillHausenblas2013}.

These advancements are significant, but they must address an important point - storage and computing are useless if \glspl{domain_expert} (i.e. not necessarily engineers) cannot use them effectively. Historically, analysts would run queries to gather data, format it, plot graphs, and write articles manually to provide insights. However, the industry is trending more towards automating some of these steps using giant SQL engines or data processing pipelines. Automation has been a significant advancement, but it brought another problem - more derived data is generated, which is very hard to keep track of.

Gathering insights from data is finally a human process, and our goal as engineers should be to assist this process by providing tooling, not to replace people with computers. In a \gls{data_mesh}, \glspl{data_lake} are connected to \glspl{data_lake}, \glspl{data_silo} are connected to other \glspl{data_silo}, comprising structured and unstructured, collected and derived, hand-picked and computer-generated data so it is becoming increasingly hard for \glspl{domain_expert} to find and manage the data sets they are interested in or to discover new ones.

Metadata, data catalogues and data indexing systems are essential tools for data-driven organisations to maintain, govern and use data in the most efficient manner possible by allowing data analysts, data scientists, domain experts and administrators to understand the sprawling \glspl{data_mesh} they are working with. Big Data systems have largely evolved without adjacent metadata management infrastructure, which can, in part, be seen as a good thing, empowering data interoperability and loosening constraints. However, as these systems mature, more advanced and complete solutions are needed to handle "Big Metadata" \cite{bigMetadataSmith2014}. As metadata and data discovery tools evolve, the need to consider the aspects of security, privacy and data access control is becoming more relevant.

\section{Technologies Used}

OpenMetadata is an open-source platform for metadata management that aims to help data-driven organisations govern their data and metadata in a flexible and extensible way, integrating with many industry-standard tools, processes and technologies. It addresses the challenges faced by organisations with large amounts of sprawling data by providing a centralised platform that handles the management, definition, storage and procurement of the metadata for many different data resources, such as tables, columns, databases, files, dashboards, charts and also for the workflows and processes that resulted in those resources. Relating to the metadata categories described above, OpenMetadata focuses on descriptive and structural but is flexible enough to accommodate any other type. The platform automates the process of metadata discovery and indexing from various sources, such as \glspl{data_lake} (e.g., Trino), databases (e.g., PostgreSQL \cite{postgresqlTech}), business intelligence tools (e.g., PowerBI \cite{powerbiTech}), messaging queues (e.g., Apache Kafka \cite{apacheKafkaTech}), workflow orchestration tools (e.g., Apache Airflow \cite{apacheAirflowTech}). OpenMetadata is designed to be API-first \cite{apiFirstBealieu} so it fully-featured support programmatic access, and provides a user-friendly interface for data consumers, metadata maintainers, and data analysts and scientists to explore data and collaborate.

Apache Ranger \cite{apacheRangerAccessControlModelGuptaMaanak2017,apacheRangerMultiLayerdGuptaMaanak2017} is an open-source security framework for enforcing fine-grained access control policies and auditing access to resources. It was developed by the Apache Foundation to integrate with systems in the Hadoop ecosystem and to tackle the security issues that commonly appear in big data applications. The framework offers a centralised point for security management across a sizeable data-driven organisation where administrators can create data access policies based on user or group permissions or data attributes (also commonly called tags) and comprehensive auditing functionality, empowering the monitoring of resource access and detection of malicious intent. Apache Ranger is widely used in healthcare 
to secure patient records \cite{rangerHealhcareRangarajan2018}, in telecommunications to protect customer information \cite{rangerTelecomAhmad2019} and in finance to comply with regulatory requirements. The framework easily integrates with other systems in the Hadoop ecosystem, such as HDFS \cite{HDFSBorthakur2008}, HBase \cite{apacheHbaseTech}, Solr \cite{apacheSolrTech} and Kafka, but also with components outside of it, such as Trino and Elasticsearch \cite{elasticsearchTech}.

\section{\label{sec:project_goal} Project Goal}

This project will explore methods of managing access control to metadata and metadata governance solutions. The work is done in the context of using OpenMetadata as the metadata store and Apache Ranger as the centralised security framework, to integrate these two systems.

We will be looking at OpenMetadata and its main functionalities, what systems for access control are already in place, what their shortcomings are and how these can be expanded. This project will also go over Apache Ranger's functionalities, how its plugins operate, and a new plugin for OpenMetadata can be added. Ranger needs to be more well documented, so this report also aims to introduce the many parts of Apache Ranger and guide to integrating services.

Finally, the project will evaluate the quality of the proposed solution, considering how secure the system is and how well it can protect users' privacy, how flexible the approach is and how easily it can be further extended and the performance impact.

\section{Current Solutions}

Open-source metadata catalogues and data discovery systems have gained traction in the previous years, as large companies such as LinkedIn, Lyft and Netflix have internal projects for the public. It is crucial to remember that these projects were created to mould around each organisation's specific workflows and pipelines; thus, these are products of their environments, and the set of features they offer reflects that. Examples of such systems include Amundsen Lyft \cite{amundsenLyftGrover2019}, LinkedIn DatHub \cite{dataHubLinkedinLan2019}, and Netflix Metacat \cite{metacatNetflixMajumdar2018}, which have similar features and functionalities compared to OpenMetadata but, just as similarly, have a lacking approach towards controlling access to metadata.

An outlier in the group of open-source solutions is Apache Atlas \cite{apacheAtlasTech}. Initially incubated by Hortonworks, it is now under the umbrella of the Apache Software Foundation, which, some may argue, positions it at a higher standard of quality compared to the former corporate-owned projects. What makes Apache Atlas an outliner is the organisation backing it but the project's attention to data governance, access control for both data and metadata and privacy of users' data. Apache Atlas is part of the Hadoop ecosystems and natively integrates with Apache Ranger, offering specialised functionality, which we will cover in later parts of the report, Apache Hive, Apache Kafka and others, but is limited in interoperability with systems outside its ecosystems. In this area, the former systems excel.

The realm of data cataloguing systems encompasses a significant portion of commercially available solutions, with Informatica \cite{informaticaTech} being a notable exemplar. However, these solutions are proprietarily licensed, exorbitantly priced, and solely accessible to large-scale organisations. Consequently, assessing their features poses a challenge.
