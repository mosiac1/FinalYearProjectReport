\chapter{Implementation}

This chapter covers the detailed implementation of all components needed for integrating Apache Ranger as an access control system for OpenMetadata. We also present engineering problems that arose during the implementation process and their solutions.

\section{Relocating the OpenMetadata Authorizer interface and related classes}

We begin by investigating the current Authorizer interface used by OpenMetadata, shown in Listing \ref{listing:original_authorizer}. The functions \mintinline{java}{listPermissions} and \mintinline{java}{getPermission} are used to query the operations a user is allowed to apply on all resource types, all resources under a specific resource type or a fully-qualified resource. The \mintinline{java}{authorize} function handles access control decisions, yielding successfully when access is authorised and throwing an \mintinline{java}{AuthorizerException} when it is not. The \mintinline{java}{authorizeAdmin} directive verifies that a user is an administrator of the system, throwing an exception otherwise. The final procedure, \mintinline{java}{decryptSecret}, is the process of being deprecated. 

The interface, along with all classes and other interfaces used in its definition, is part of the \mintinline{text}{org.open-metadata:openmetadata-serivce} submodule and is tightly bound to OpenMetadata's API service implementation. This is an issue, as the Apache Ranger authoriser implementation needs to be isolated in a separate submodule that does not depend on \mintinline{java}{openmetadata-service}. The solution proposed here is to relocate the \mintinline{java}{Authorizer} interface, with related classes and interfaces, to a new submodule \mintinline{text}{org.open-metadata:openmetadata-authorization}; thus, the API service implementation submodule and Apache Ranger authoriser submodule may then depend on it.

We define that \mintinline{text}{org.open-metadata:openmetadata-authorization} submodule to depend on the \mintinline{text}{org.open-metadata:openmetadata-spec} module, providing entity definitions, and to contain four interfaces under the \mintinline{java}{org.openmetadata.security} package: \mintinline{java}{Authorizer}, \mintinline{java}{OperationContextInterface}, \mintinline{java}{ResourceContextInterface}, \mintinline{java}{SecurityContextInterface} and the \mintinline{java}{AuthorizationException} class. The \mintinline{java}{Authorizer} interface is updated to use the new interfaces.

\begin{listing}

\begin{minted}[breaklines,
    linenos,
    fontsize=\footnotesize,
    frame=single,
]{java}
public interface Authorizer {
    void init(OpenMetadataApplicationConfig openMetadataApplicationConfig, Jdbi jdbi);

    List<ResourcePermission> listPermissions(SecurityContext securityContext, String user);

    ResourcePermission getPermission(SecurityContext securityContext, String user, String resource);

    ResourcePermission getPermission(
        SecurityContext securityContext, String user, ResourceContextInterface resourceContext);

    void authorize(
        SecurityContext securityContext, OperationContext operationContext, ResourceContextInterface resourceContext)
        throws IOException;

    void authorizeAdmin(SecurityContext securityContext);

    boolean decryptSecret(SecurityContext securityContext);
}
\end{minted}

\caption{Original \mintinline{java}{org.openmetadata.service.security.Authorizer} interface in the OpenMetadata source code.}
\label{listing:original_authorizer}

\end{listing}

\subsection{Initialisation}

The \mintinline{java}{OpenMetadataApplicationConfig} and \mintinline{java}{Jdbi} classes are part of the \mintinline[breaklines, breakafter=.:]{java}{org.open-metadata:openmetadata-serivce} module can not be used in the authorisation interface.

OpenMetadata handles the initialisation of resources in the \mintinline{java}{OpenMetadataApplication} class. The authoriser instance is created based on the \mintinline{text}{authorizerConfiguration.className} property provided by the application configuration. The following expression handles class loading and instance creation, the configuration properties being passed later to the \mintinline{java}{init} function.

\begin{minted}[breaklines, breakbefore=.]{java}
    this.authorizer = Class.forName(authorizerConf.getClassName()).asSubclass(Authorizer.class).getConstructor().newInstance();
\end{minted}

The initialisation sequence is updated to use the class constructor that takes a single argument, an instance of \mintinline{java}{AuthorizerConfiguration}, and the \mintinline{java}{init} function is modified to take no arguments.

\begin{minted}[breaklines, breakbefore=.]{java}
    this.authorizer = Class.forName(authorizerConf.getClassName()).asSubclass(Authorizer.class).getConstructor(AuthorizerConfiguration.class).newInstance(authorizerConf);
\end{minted}

The provided authoriser implementations, \mintinline{java}{DefaultAuthorizer} and \mintinline{java}{NoopAuthorizer}, are updated to follow the revised structure. Listing \ref{listing:new_authorizer} provides the final layout of the authorisation interface that the Apache Ranger authoriser will implement.

\begin{listing}

\begin{minted}[breaklines,
    linenos,
    fontsize=\footnotesize,
    frame=single,
]{java}
public interface Authorizer {
    default void init() {}

    List<ResourcePermission> listPermissions(SecurityContextInterface securityContext, String user);

    ResourcePermission getPermission(SecurityContextInterface securityContext, String user, String resource);

    ResourcePermission getPermission(
        SecurityContextInterface securityContext, String user, ResourceContextInterface resourceContext);

    void authorize(
        SecurityContextInterface securityContext,
        OperationContextInterface operationContext,
        ResourceContextInterface resourceContext)
        throws IOException;

    void authorizeAdmin(SecurityContextInterface securityContext);

    boolean decryptSecret(SecurityContextInterface securityContext);
}
\end{minted}

\caption{Updated \mintinline{java}{org.openmetadata.service.Authorizer} interface in the OpenMetadata source code.}
\label{listing:new_authorizer}

\end{listing}

\section{Using Apache Ranger libraries for authorisation}

This section details the initialisation and utilisation of Apache Ranger libraries for access control policy evaluation through the \mintinline{java}{RangerBasePlugin} class. It also touches on using the Hadoop user group information utility to retrieve users' group membership data for a centralised Active Directory database.

\subsection{Initialisation of the Ranger Plugin}

The Ranger base plugin loads its configuration from XML files structured as lists of key and value string pairs. This behaviour is handled by the \mintinline{java}{RangerPluginConfig} class that extends the commonly used Hadoop class \mintinline{java}{Configuration}.

Apache Ranger's configuration loading utility expects properties to be provided in three files, \mintinline{text}{ranger-<serviceType>-audit.xml}, \mintinline{text}{ranger-<serviceType>-security.xml} and\\ \mintinline{text}{ranger-<serviceType>-policymgr-ssl.xml} (where \mintinline{text}{<serviceType>} is a placeholder for the type of the service that the plugin is being initialised for, in this case taking the value \mintinline{text}{openmetadata}), which it will attempt to read from the current directory. We extend this functionality by allowing administrators to specify alternative locations for these files in OpenMetadata's configuration. The Hadoop utility for querying user group information is initialised similarly. Sample configuration files are in the OpenMetadata source code under the \mintinline{text}{ranger/openmetadata-authorization-ranger/conf} folder.

\section{Utilising the Ranger Plugin}

The Ranger plugin evaluates access requests using the function \mintinline{java}{isAccessAllowed}, which takes a single argument, an instance of the \mintinline{java}{RangerAccessRequest} class which wraps identification of the user requesting access (consisting of the user's unique name and the groups and roles it is a member of), a description of the resource being accessed, and the operation applied to the resource. 

User identification is handled by the mintinline{java}{RangerHadoopUserGroupsRolesProvider} class. Resolving the need data begins with the \mintinline{java}{SecurityContextInterface}, which supplies the user's name. The Ranger Authorizer uses the Hadoop User Group Information utility to get the user's groups.

\begin{minted}[breaklines, breakbefore=.]{java}
Set<String> groups = UserGroupInformation.createRemoteUser(internalName).getGroupNames();
\end{minted}

The Ranger Plugin is then used to retrieve the appropriate roles based on the user's name and the groups.

\begin{minted}[breaklines, breakbefore=.]{java}
Set<String> roles = rangerPlugin.getRolesFromUserAndGroups(user, groups);
\end{minted}

The Ranger Plugin handles references to resources as a mapping between string keys, representing the resource type, and generic objects, representing the value of the resource type, through the \mintinline{java}{RangerAccessResourceImpl} class. The generic objects used for resource values are commonly strings or collections of strings. The \mintinline{java}{RangerOpenmetadataAccessResource} class extends \mintinline{java}{RangerAccessResourceImpl}, implementing the mapping of OpenMetadata entities to key-values pairs recognised by the Ranger Plugin. The class handles setting values for entities with parents by reflectively invoking methods that take certain entity types as arguments, as depicted in Listing \ref{listing:ranger_openmetadta_access_resource}.

\begin{listing}

\begin{minted}[breaklines,
    linenos,
    fontsize=\footnotesize,
    frame=single,
]{java}
public class RangerOpenmetadataAccessResource extends RangerAccessResourceImpl {
    public RangerOpenmetadataAccessResource(ResourceContextInterface resourceContextInterface) {
        ...
        EntityInterface entity = resourceContextInterface.getEntity();
        try {
            Method method = this.getClass().getDeclaredMethod("setSubValues", entity.getClass());
            method.invoke(this, entity);
        } 
        ...
    }
    
    private void setSubValues(Table table) {
        setValue(table.getService(), DatabaseService.class);
        setValue(table.getDatabase(), Database.class);
        setValue(table.getDatabaseSchema(), DatabaseSchema.class);
    }
    
    ...

    private void setValue(EntityReference entityReference, Class<? extends EntityInterface> clazz) {
        if (entityReference == null) {
            setValue(formatOpenMetadataClassNameToRangerResource(clazz), "*");
        } else {
            setValue(entityReference);
            }
    }

    private void setValue(EntityReference entityReference) {
        setValue(getRangerResourceName(entityReference), entityReference.getName());
    }
}
\end{minted}

\caption{Snippet of the \mintinline{java}{RangerOpenmetadataAccessResource}, using reflection to set Ranger Access Resource values from OpenMetadata entities.}

\label{listing:ranger_openmetadta_access_resource}

\end{listing}

The access type is passed to the Ranger Plugin as a string, retrieved from the \mintinline[breaklines, breakbefore=CI]{java}{OperationContextInterface}. As this interface provides a list of operations, each is checked independently.

Listing \ref{listing:ranger_authorizer_impl_authorize} supplies the body of the \mintinline{java}{authorize} function provided by the Ranger Authorizer implementation of the OpenMetadata \mintinline{java}{Authorizer} interface. The procedure uses the \mintinline{java}{RangerOpenmetadataAccessRequest} class, which extends the provided request class \mintinline[breaklines, breakbefore=ARI]{java}{RangerAccessRequestImpl}, adding convenience constructors. The \mintinline{java}{isAccessAllowed} procedure of the Ranger Plugin returns a \mintinline{java}{RangerAccessResult} object, which contains the final allowed or denied decision along with meta information of the policy evaluations.

\begin{listing}

\begin{minted}[breaklines,
    linenos,
    fontsize=\footnotesize,
    frame=single,
]{java}
public void authorize(
        SecurityContextInterface securityContext,
        OperationContextInterface operationContext,
        ResourceContextInterface resourceContext)
        throws IOException {
    RangerUserGroupsRoles ugr = groupsRolesProvider.getUserGroupsRoles(securityContext);
    RangerAccessResource accessResource = new RangerOpenmetadataAccessResource(resourceContext);
    for (MetadataOperation operation : operationContext.getOperations()) {
        RangerAccessRequest req = new RangerOpenmetadataAccessRequest(accessResource, operation, ugr);
        RangerAccessResult result = rangerPlugin.isAccessAllowed(req);
        if (result != null && !result.getIsAllowed()) {
            throw new AuthorizationException(String.format("[RangerAccessRequest=%s] Permission denied by Ranger: %s", req, result));
        }
    }
}
\end{minted}

\caption{Snippet of \mintinline{java}{RangerAuthorizerImpl}, displaying the implementation of the authorisation check.}
\label{listing:ranger_authorizer_impl_authorize}
    
\end{listing}

\subsection{Listing user permissions for resources}

The Ranger Plugin enables queries of access control rules (commonly abbreviated to ACLs) with the procedure \mintinline{java}{getResourceACLs}, which accepts a single argument, a \mintinline[breaklines, breakbefore=AR]{java}{RangerAccessRequest} object. Creating an access request is the same as detailed above. An additional case appears when querying permissions for all resources of a type, which is handled by mapping, in the \mintinline{java}{RangerAccessResourceImpl}, the resource type and its ancestor to wildcards (the asterisk symbol).

The \mintinline{java}{getResourceACLs} function returns a \mintinline{java}{RangerResourceACLs} object containing three maps of the type \mintinline{java}{Map<String, Map<String, AccessResult>>}, one for each of user, group and role access control rules. Keys of the outer map represent the name of a user, group or role, respectively, while keys of the inner map represent access type, and values of the inner map represent an access result, which can be allowed, denied or undetermined. This response is not aggregated, as it will contain results for all users, groups and roles present in policies that match the resource queried and outcomes for all access types, not only those applicable to the resource type used in the request. Results are consolidated by the \mintinline{java}{RangerPermissionsProvider} class, filtering groups and roles the user is not part of and access types which do not correspond to the resource type in question, and combines the remaining results. An access type is considered permitted if any of the user's groups for roles allow it or if the user is specifically allowed that access type.